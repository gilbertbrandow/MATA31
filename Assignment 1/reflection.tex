\documentclass{article}
\usepackage[utf8]{inputenc}
\setlength{\parindent}{0pt}
\usepackage[english]{babel}
\usepackage{csquotes} 
\usepackage{amssymb}
\setlength{\parindent}{0pt}

\title{3. Reflection on Learning}
\author{Simon Gustafsson}
\date{15th of September 2024}

\begin{document}

\maketitle

The AI influenced my understanding of the problem by providing helpful initial hints, such as defining the set \( S \). Its explanation of why \( k_0 - 1 \leq x \) was particularly valuable, as it clarified how I could apply the Well-ordering Principle to deduce it.

Working with AI for my first time writing in LaTeX was beneficial. LaTeX can be intimidating for beginners due to its syntax and the technicalities of formatting equations and structuring documents. The AI acted as a helpful guide, explaining how to correctly use LaTeX commands and providing clear examples for formatting mathematical notation, such as defining sets. This not only saved time but also helped me quickly grasp how to write proofs clearly in LaTeX. The AI's ability to suggest best practices for structuring sections, labeling, and citing references ensured that my document was well-organized and easy to read, which I might have struggled with independently.

However, there were several moments where the AI provided too much of the solution upfront, which detracted from the learning experience. I often had to ask the AI to refrain from giving the full solution so I could work through the details myself and even still it gave away quite a bit. This aspect of working with AI was a bit frustrating. In future interactions, I will in every prompt be more specific about the level of guidance I need.

Going forward, I see myself using AI as a tool for converting my mathematical ideas and notes into LaTeX efficiently. The AI helps streamline the process of formatting complex equations and structuring proofs, which saves time and reduces errors. Additionally, I do find it valuable for discussing potential solutions, as it can offer useful perspectives that can clarify difficult concepts or suggest alternative approaches to problems.

\end{document}
