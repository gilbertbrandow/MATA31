\documentclass{article}
\usepackage[utf8]{inputenc}
\usepackage[english]{babel}
\usepackage{csquotes} 
\usepackage{amssymb}
\usepackage[backend=biber,style=alphabetic]{biblatex}
\newtheorem{lemma}{Lemma}
\addbibresource{references.bib}
\setlength{\parindent}{0pt}

\title{Applying the Well-Ordering Principle: Integer Intervals for Real Numbers}
\author{Simon Gustafsson}
\date{14th of September 2024}

\begin{document}
\maketitle

\section*{Introduction}
This proof addresses **Exercise 1.122** from the textbook *Don't Panic: A Guide to MATA21 Analysis in One Variable* \cite[pg. 41]{dontpanic}.

\section*{Problem Statement}

\textbf{Assumptions:} Let \( x \in [0, \infty) \) and \( n \in \mathbb{Z} \). \\
\textbf{Statement:} For every \( x \), there exists an \( n \) such that \( x \in [n, n+1) \).

\section*{The Well-Ordering Principle}
\begin{lemma}
Every non-empty subset of the natural numbers \( \mathbb{N} \) has a least element \cite[pg. 41]{dontpanic}.
\end{lemma}

\section*{Archimedean Property}
The Archimedean Property states that for any real number \( x \in \mathbb{R} \), there exists a natural number \( n \in \mathbb{N} \) such that \( n > x \) \cite[pg. 38]{dontpanic}.

\section*{Proof}

Since \( x \geq 0 \), any integer \( k \) that satisfies \( k > x \) must be positive or zero. Therefore, we restrict \( k \) to be in \( \mathbb{N} \), the set of natural numbers. Now, consider the set
\[
S = \{ k \in \mathbb{N} \mid k > x \}.
\]
\vspace{-0.5\baselineskip}

By the Archimedean Property (see previous section), for any real number \( y \geq 0 \), there exists a natural number \( m \in \mathbb{N} \) such that \( m > y \). Therefore, the set \( S \) is non-empty. Since \( S \) is a non-empty subset of \( \mathbb{N} \), the Well-Ordering Principle guarantees the existence of a least element. Denote this smallest element by \( k_0 \). By the definition of \( S \), we know that
\[
k_0 > x.
\]
\vspace{-0.5\baselineskip}

Since \( k_0 \) is the smallest natural number such that \( k_0 > x \), it follows that \( k_0 - 1 \) cannot be greater than \( x \). If \( k_0 - 1 > x \), then \( k_0 - 1 \) would also belong to the set \( S \), contradicting the fact that \( k_0 \) is the smallest element of \( S \). Thus, we conclude that
\[
    k_0 - 1 \leq x.
\]
\vspace{-0.5\baselineskip}

Now let
\[
n = k_0 - 1.
\]
\vspace{-0.5\baselineskip}

From the previous steps, we know that \( k_0 > x \) and \( k_0 - 1 \leq x \), therefore
\[
n \leq x < n + 1.
\]
\vspace{-0.5\baselineskip}

Thus, for every \( x \geq 0 \), there exists an integer \( n \in \mathbb{N} \) such that \( n \leq x < n + 1 \). Since \( \mathbb{N} \subseteq \mathbb{Z} \), we can conclude that for every \( x \geq 0 \), there exists an integer \( n \in \mathbb{Z} \) such that \( x \in [n, n+1) \), as required.

\printbibliography
\end{document}
